\documentclass{mini}
\usepackage[utf8]{inputenc}
\usepackage{caption}
\usepackage{subcaption}
\usepackage[polish]{babel}
\usepackage{graphicx}
\usepackage{mathtools}
\usepackage{algpseudocode}
\usepackage{color}
\usepackage{xcolor}
\usepackage{listings}
\usepackage{catchfilebetweentags}
\usepackage{enumitem}

\usepackage{catchfilebetweentags}
\usepackage{etoolbox}
\makeatletter
\patchcmd{\CatchFBT@Fin@l}{\endlinechar\m@ne}{}
  {}{\typeout{Unsuccessful patch!}}
\makeatother

\addto\extraspolish{%  
 \def\figureautorefname{Rysunek}%  
} 

%------------------------------------------------------------------------------%
\title{Realizacje scenariuszy działań}
\author{Robert Jakubowski
\\Hanna Dziegciar
\\Paweł Bielicki
\\Karol Bocian
\\Karol Dzitkowski
\\Mateusz Jankowski
\\Wiktor Ryciuk
\\Mariusz Ambroziak}
\monthyear{\today}
%------------------------------------------------------------------------------%
\begin{document}
%<*tag>

\section{Opis języka akcji}
Język akcji zaprojektowany na potrzeby zadania, musi spełniać następujące warunki:
\begin{enumerate}
	\item Prawo inercji.
	\item Sekwencyjność działań.
	\item Możliwe akcje niedeterministyczne.
	\item Liniowy model czasu - czas dyskretny.
	\item Pełna informacja o wszystkich:
	\begin{enumerate}
		\item akcjach,
		\item skutkach bezpośrednich.
	\end{enumerate}
	\item Akcja posiada:
	\begin{enumerate}
		\item warunek początkowy,
		\item czas trwania $ t \geqslant 1, t\in\mathbb{N} $,
		\item efekt akcji.
	\end{enumerate}
	\item Podczas trwania akcji, wartości zmiennych, na które ona wpływa, nie są znane.
	\item Występujące rodzaje efektów:
	\begin{enumerate}
		\item środowiskowe,
		\item dynamiczne.
	\end{enumerate}
	\item Akcje mogą być niewykonalne.
	\item Stany opisywane częściowo (obserwacje). (TODO wyjaśnić) %po wyjaśnieniu, usunąć TODO
	\item Pewne stany mogą rozpocząć wykonywanie pewnych akcji.
\end{enumerate}

Językiem odpowiadającym powyższym założeniom jest język $ \textit{AL} $ opisujący domeny akcji z czasem liniowym.

\subsection{Sygnatura języka}
$ \psi = (F, Ac, \mathbb{N}) $\\
gdzie:\\
$ F $ -- zbiór zmiennych inercji (fluentów)\\
$ Ac $ -- zbiór akcji\\
$ \mathbb{N} $ -- zbiór liczb naturalnych (czas trwania akcji)

\subsection{Opis domeny}
Rodzaje zdań występujących w projektowanym języku (domena języka):\\
Oznaczenia:\\
$ f $ -- fluent\\
$ Ac_{i}, Ac_{j}\in Ac $\\
$ \pi \in Forms(F) $\\
$ d_{i}, d \in \mathbb{N} $
\begin{itemize}
	\item initially $ \alpha $\\
	Określa stan początkowy fluentów w formule $ \alpha $.
	\item $ (Ac_{i}, d_{i}) $ causes $ \alpha $ if $ \pi $\\
	Akcja $ Ac_{i} $ trwająca $ d_{i} $ chwil powoduje stan $ \alpha $, jeśli zachodzi warunek $ \pi $.
	\item $ (Ac_{i}, d_{i}) $ invokes $ (Ac_{j}, d_{j}) $ after $ d $ if $ \pi $\\
	Akcja $ Ac_{i} $ trwająca $ d_{i} $ chwil powoduje wykonanie akcji $ Ac_{j} $ trwającej $ d_{j} $ chwil po $ d $ chwilach od zakończenia akcji $ Ac_{i} $, jeśli zachodzi warunek $ \pi $.
	\item $ (Ac_{i}, d_{i}) $ releases $ f $ if $ \pi $\\
	Akcja $ Ac_{i} $ trwająca $ d_{i} $ chwil powoduje uwolnienie $ f $ po zakończeniu akcji $ Ac_{i} $, jeśli zachodzi warunek $ \pi $.
	\item $ \pi $ triggers $ (Ac_{i}, d_{i}) $\\
	Akcja $ Ac_{i} $ trwająca $ d_{i} $ chwil jest wykonywana, jeśli zajdzie warunek $ \pi $.
\end{itemize}

\subsection{Scenariusze działań}
Scenariusze działań opisane są w następujący sposób:
	\begin{itemize}
	\item $ Sc = (OBS, ACS) $
	\item $ OBS = \{(\gamma_{1}, t_{1}),...,(\gamma_{m}, t_{m})\} $, gdzie:\\
	$ m \geqslant 0 $ -- obserwacje, gdzie każda obserwacja jest stanem częściowym (stanem spełniającym warunek $ \gamma $ w pewnym punkcie czasu $ t $).\\
	$ \gamma $ -- zbiór (np. $ x_{1}=True, x_{2}=True, x_{3}=False $).
	\item $ ACS = \{((Ac_{1},d_{1}), t_{1}),...,((Ac_{n},d_{n}), t_{n})\} $, gdzie:\\
	$ n \geqslant 1 $,\\
	$ Ac_{i} $ -- akcja,\\
	$ d_{i} $ -- czas trwania akcji,\\
	$ t_{i} $ -- punkt w czasie (rozpoczęcie akcji).
\end{itemize}

\subsection{Semantyka}
\begin{definition}Semantyczną strukturą języka $ \textit{AL} $ nazywamy system $ S=(H,O,E) $  taki, że:
\begin{itemize}
\item $ H: F $ $\times$ $ \mathbb{N}$ $\longrightarrow$ $\{0,1\}$ jest funkcją historii, pozwala ona stwierdzić, jaki stan ma pewny fluent w danej chwili czasu.
\item $ O: Ac$ $\times$ $ \mathbb{N}$ $\longrightarrow$ $2^{F}$ jest funkcją okluzji. Dla pewnej ustalonej akcji $A$ i chwili czasu $t\in\mathbb{N}$ funkcja $O(A,t) $ zwraca zbiór fluentów, na który akcja $A$ ma wpływ, jeśli zostanie wykonana od czasu $t-1$ do $t$.
\item $E\subseteq Ac \times \mathbb{N}$ jest relacją wykonań akcji. Para $(A,t)$ należy do relacji $E$ jeśli akcja $A$ jest wykonana w czasie t. W naszym modelu zakładamy warunek sekwencyjności działań. Oznacza on, że tylko jedną akcje możemy wykonać w danym czasie tak, więc jeśli $(A,t)\in E$ oraz $(B,t)\in E$, to $A=B$. 
\end{itemize}
\end{definition}

 Niech: $A,B$ będą akcjami, $ f $ - fluentem, $\alpha, \pi$ - literałami, $d $ - liczbą naturalną oraz $fl(\alpha)$ będzie zbiorem fluentów występujących w $\alpha$. Wtedy dla zdań języka $AL$ muszą być spełnione następujące warunki: 
   \begin{itemize} 
   \item Dla każdego wyrażenia $ (A\;causes\;\alpha\;if\;\pi)\in D $ i dla każdego momentu w czasie $t \in \mathbb{N}$, jęzeli $H(\pi,t)=1$ oraz $(A,t) \in E$, wtedy $H(\alpha,t+1)=1$ i $fl(\alpha)\subseteq O(A,t+1)$.
   \item Dla każdego wyrażenia $(A\;release\;f\;if\;\pi)\in D $ i dla każdego momentu czasu $t \in \mathbb{N}$, jeżeli $H(\pi,t)=1$ oraz $(A,t)\in E$, wtedy $f\in O(A,t+1)$.
   \item Dla każdego wyrażenia $(\pi\;triggers\;A)\in D$ i dla każdego momentu czasu $t \in \mathbb{N}$, jeżeli $H(\pi,t)=1$, wtedy $(A,t)\in E$.
   \item Dla każdego wyrażenia $(A\;invokes\;B\;after\;d\;\;if\;\pi)\in D$ i dla każdego momentu czasu $t \in \mathbb{N}$, jeżeli $H(\pi,t)=1$ oraz $(A,t)\in E $, wtedy $(B,t+d+1) \in E $.
   \end{itemize}
   \begin{definition}
   Niech $S = (H,O,E)$ będzie strukturą języka $ \textit{AL} $, $ Sc=(OBS,ACS) $ będzie scenariuszem, oraz $ D $ domeną. Powiem, że $ S $ jest strukturą dla $Sc$ zgodnym z opisem domeny $D$ jeśli:
     \begin{itemize} 
     \item Dla każdej obserwacji $(\alpha,t )\in OBS$, $H(\alpha,t )=1$
     \item $ACS \subseteq E$ 
    \end{itemize} 
   \end{definition}
   \begin{definition}
   Niech $O_{1}$,$O_{2}$: $X \longrightarrow 2^{Y}$, mówimy, że $O_{1} \prec O_{2}$ jeżeli $\forall x\in X$ $O_{1}(x)\subseteq O_{2}(x)$ oraz $O_{1}\neq O_{2}$.
   \end{definition}
   
   \begin{definition}
   Niech $S=(H,O,E)$ będzie strukturą dla scenariusza $Sc=(OBS,ACS)$ zgodną z opisem domeny $D$. Mowimy, że $S$ jest $O$-minimalną strukturą, jeżeli nie istnieje struktura $S'=(H',O',E')$ dla tego samego scenariusza i domeny taka, że $O'\prec O$.  
   \end{definition}
   \begin{definition}
   Niech $S=(H,O,E)$ będzie strukturą dla scenariusza $Sc=(OBS,ACS)$ zgodną z opisem domeny $D$. $S$ będziemy nazywać modelem $Sc$ zgodnym z opisem $D$ jeżeli:
   \begin{itemize}
   \item $S$ jest $O$-minimalny
   \item Dla każdego momentu w czasie $t\in \mathbb{N}$, ${f\in F:H(f,t)\neq H(f,t+1)}\subseteq O(A,t+1)$ dla pewnej akcji $A\in Ac$.
   \item Nie istnieje, żadna struktura $S'=(H',O',E')$ dla $Sc$ zgodna z opisem $D$ która spełnia poprzednie warunki oraz taka, że $E'\subset E$. 
   \end{itemize}
   \end{definition}
   \begin{remark}
   Nie dla każdego scenariusza można ułożyć model. Mówimy, że scenariusz $Sc$ jest \textit{zgodny} jeśli istnieje do niego model zgodny z domeną $D$.
   \end{remark}

%</tag>
\end{document}
